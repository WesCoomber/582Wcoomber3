\section{implementation}
\label{sec:impl}
Our system abstracts the idea of speculative execution to the more general idea
of {\it pre-works}. Across many domain spaces, speculative execution can entail many
different things such as pre-fetching from hard disk to memory, pre-fetching
the results of a predicted query from a remote web server, and pre-computing
the results of a predicted query on local data. This causes unnecessary
complication and complexity for the application developers who are interested in
using excess resources to do some work ahead-of-time. Pre-works are a
higher-level concept that encompasses all of the aforementioned speculative
execution use-cases and also considers more abstract concepts such as the
`potential reusability' of pre-works and the system's current excess
resources. By thinking of pre-works as a tool to trade excess resources for
improved responsiveness and improved user-experience, we can more easily apply
speculative execution to many different applications.

Due to the limited time-constraints of this class, our class project had
limited scope with regards to the original higher-level idea of `Speculation
on Steroids'. One limitation of our current work is that the speculation only
happens with our single implemented application. One of the core-motivations
behind the project, was to create a generic speculation framework that
provides significant benefit across many diverse applications. To have more
evidence of our system providing this `general speculation framework', we
would like to test our system across multiple applications and then benchmark
the performance on each application. 

Another limitation of the project was the difficulty in finding real data sets
of user’s traces through an application. We spent the first half of the
semester trying to find public and/or academic datasets from Google, Facebook,
and Microsoft to no avail. Ultimately, we settled on using the academic Yelp
data-set as a candidate application for generalized speculation, but because
we didn’t have exact user-traces through their application, we still
encountered some issues with this approach. 

We ended up having to utilize an analogous heuristic composed of the
restaurant’s ratings and common reviewers shared between restaurants-- instead
of exact user’s paths through the program. This let us implement an interactive
system in time for the class, but also limits the conclusions we can derive from
the system because it is another level of abstraction from the true generalized
speculation based on user-traces that our original idea proposed.

