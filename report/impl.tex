\section{Design}
\label{sec:design}
In this section, we explain and justify design choices, and consider inherent
limitations in our design choices.

% \subsection{Steroid}
We believe that steroid - bi-conditional probability tree - must be as generic
as applicable to an arbitary set of data, programs, and users; 
however, due to the scope of this project and limited time constrained fractors,
we have only implemented rank-based steroid, that considers only top N choices.

\subsection{Bi-conditional probability tree}
Bi-conditional by definition takes account of ``if this, and that'' model. 
That is, given current job initiated by user input (if this), find the probability of next
step(s) (and that). 
We believe that bi-conditional probability neatly fits into our need as we aim
to predict next steps from present step, not all possible steps.
Furthermore, traces of steps taken by a user influences the prediction of next
run by another user.

In other words, bi-conditional probability reveals a trend over series of steps
by users and outputs the best candidate of next step(s) by current user by
observing 1) current and previous step(s), and 2) other users' traces, which
is why we note as `bi-'.
For instance, there are two distinct graduate school committee members - Alice
and Bob - attempt to list the best candidates with different criteria.  If Alice
so far sorted the columns by last name, and if Bob already finished his
reviewing by sorting students by first and last name, GRE scores, our steroid
predicts that Alice has a high probability of sorting students by GRE scroes in
the future.  From this steroid, our framework attempts to pre-work
asynchoronously (sorting columns by GRE scores in this example) while Alice is
still looking over the list of students sorted by last name. 

We have found that tree-like structure is a perfect match for user-facing
application and to build steroids. So, on top of distributin of bi-conditional
probability, we have built tree structure, in which each node represents an
action initiated by a user input and maintains prior probability distribution
that led to the node. 

Tree-like structure gives another benefit: a rollback recovery. Often, user
mis-clicks certain inputs and has to wait for significant time, as the work
already has been initated, in order to rollback to previous state. 
Tree-like structure in our framework maintains shallow copy for each nodes that
allows fast recovery to previous state if 1) user unintentionally enters wrong
input, and 2) when speculation is missed. 

The former case may be an application-specific as a few applications already
handle this case by prompting confirm button or aborts the mis-entered job.
However, the later case without tree-like structure not only imposes a hard
penality by wasting resources in mis-speculated jobs but also requires to
re-work all the steps taken by a user from entry to one before mis-speculation.
Tree-like strcuture provides easy rollback recovery with negligible cost in
terms of resources and latency.

\subsection{Limitation}
The drawback of bi-conditional probability tree is that in order to build
precise enough tree,  large volume of user traces in similar category of
applications or users interacting with similar dataset is necessary.
As explained in later section~\ref{sec:impl}, we have not found publicly
available users traces that we can construct the tree with high speculation hit.

This limitation also hinders our framework in terms of performance: because the
large of volume of user traces needs to be analyzed at runtime, even single miss
in speculation exceeds the latency improvement by pre-works. 
As discussed in section~\ref{sec:future}, we plan to investigate a way to
transform tree builiding in offline and even escalate the absraction of steroid
to higher level, such that application domain-specific knowledge acts a
``tunning'' to improve the hit ratio in speculation.

% Joseph: simply remove below section if time runs out...
\section{Implementation}
\label{sec:impl}
In this section, we discuss current implementation in details.
We have implemented somewhat data-specific steroid due to the scope of this
project, time constrained factors, and most importantly, lack of real-world
user traces as mentioned in previous section. 

The steroid built for Yelp dataset takes the following heuristics to calculate
the probability of each restaurants selected by an user: user-rating, stars,
review counts, and common subset users who reviewed restaurant in common and also
reviewed another restaurant. 

After builing the steroid, we have speculation phase that for each of user
selection, outputs distribution of probability in next steps and executes works
required in likely future steps. 
This is very similar to MapReduce frameworks, but we believe that it is
different as there is only one selection available in analogy to Map phase. 



