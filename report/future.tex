\section{Future Works}
\label{sec:future}

Future work to be done in the project is to apply the system to multiple applications across a diverse domain space. One of the key limitations of project due to time-constraints and limited sources of data and/or user-traces, was that our implemented system only works in the single Yelp restaurant application. Our proposed generalized system is intended to provide a general framework for speculative 'pre-work' that significantly improves user-perceived latency across many different applications. To confirm that our system does effectively generalize across many applications, we would like to implement Pythia on a series of applications and then benchmark their performance, resource usage, and user-perceived latency for the interative applications. 

Another idea worth exploring is using machine learning or iterative Hidden Markov Machines (HMMS) to generate models of weighted probabilistic user-trace paths through the applications. This would let our system dynamically respond to the common user's paths through the application and change accordingly. We would have to make sure we have a significant amount of training data and test data to deal with overfitting issues with the speculative execution. Finally the other future work to be investigated is the greater practicality and efficiency of re-using mispredicted work. We didn't get to implementing re-use of mispredicted work in our project, but we think it is a promising feature based on prior works. By implementing the recycling of mis-predicted and traditionally unused pre-computation, we could more accurately evaluate the benefits and trade-offs of the approach. If the results are promising, and re-using mispredicted work is viable in most applications, then we would introduce that factor into our weighted decision/utility function that determines which work we speculate upon.