\begin{abstract}
\label{sec:abs}
Traditional approach in hiding computational, I/O fetching or ``work'' latency
is by simplifying the user-interface so that for each of an input by a user
results in as small user-interactive work as possible. 
This approach in hiding latency is widely adapted in user-interactive system
such as mobile systems where user-experience is mainly ranked by visible work
latency.
However, the fundamental drawback in big-data analysis -- equipped with high
computation, I/O, network latency --  presents a challenge for the existing
approach. 

As the volume of data located locally or in cloud exponentially grows, even a
simple action by a user such as {\it sorting} in alphabetical order takes an
order of minutes. 
We expect this legging delay (work latency) between input by a user and output
to user will increase in the era of Bigdata and Internet of Things where
Terabytes of data are produced in every second. 
In this report, we can reduce the work latency by ``speculating'' next steps
that will be taken by a user in response to the output by given user inputs. 
We achieve this by building statistical distribution called ``steroid'' that
maintains individual probability of next possible inputs by user when output of
current work is present to the user.

We generalize this notion of {\it speculation on steroid} that develops online
learning model at runtime in order to speculate which work is most likely to be
executed by a user in next several program flows, and does ``pre-work'' required
by next flows whilst waiting for completion of user-interacting tasks such as
requesting an input, showing pictures.  
We design, implement, and evaluate novel framework called {\it Pythia}, which
dynamically builds steroid in a form of bi-conditional probability tree that is
dataset, program agnostic. 
Toward the end, our prototype with Yelp dataset shows that by correctly,
statistically speculating future non-deterministic program flows, {\it Pythia}
improves 6x latency of fetching desired pictures from cloud datastore. 
\end{abstract}
